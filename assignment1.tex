\documentclass[a4paper]{article}
\usepackage[T1]{fontenc}
\usepackage[utf8]{inputenc}
\usepackage{lmodern}
\usepackage{amsmath}
\usepackage{amsfonts}
\usepackage{graphicx,times}
\usepackage{url,paralist,xspace,textcomp,fullpage}

\newcommand{\west}{\ensuremath{\textbf{West}}\xspace}
\newcommand{\east}{\ensuremath{\textbf{East}}\xspace}
\newcommand{\south}{\ensuremath{\textbf{South}}\xspace}
\newcommand{\north}{\ensuremath{\textbf{North}}\xspace}
\newcommand{\length}{\ensuremath{\textbf{length}}\xspace}
\newcommand{\el}{\ensuremath{\textbf{`elem`}}\xspace}
\newcommand{\suc}{\textbf{\ttfamily succ}\xspace}
\newcommand{\prd}{\textbf{\ttfamily pred}\xspace}

\newcommand{\src}[1]{\texttt{#1}\xspace}
\newcommand{\func}[1]{\textbf{\ttfamily #1}\xspace}
\newcommand{\functype}[2]{\mbox{\texttt{\scriptsize\textbf{#1} ::\ #2}}\xspace}

\title{Advanced Programming\\
Assignment 1: Navigating the Maze}
\author{Thomas Bracht Laumann Jespersen\\ \url{ntl316@alumni.ku.dk} \and Marco Eilers\\ \url{dbk726@alumni.ku.dk} }


\usepackage{listings}
\lstset{basicstyle=\ttfamily\scriptsize}


\begin{document}
\maketitle


\section{Explanation of the code}
The module \src{World} contains all functions and types only concerning the maze, i.e. directions, positions and functions for manipulating them, as well as the testMaze specified in the assignment. 

\src{MEL.hs} contains all types and functions concerning the interpreter and \src{Test.hs} contains several kinds of unit- and integration tests concerning both the maze and the MEL interpreter.

\subsection{The Maze}
Our type for the maze is as follows:
\begin{lstlisting}[language=haskell]
data Maze = Maze { width  :: Int
                 , height :: Int
                 , cells  :: M.Map Position Cell
                 } deriving (Show)
\end{lstlisting}

We require that the maze fulfills the following constraints:  

\begin{itemize}
  \item It must not be possible to leave the maze:
    \begin{itemize}\setlength{\itemsep}{-2pt}
      \item \verb|West `elem` cells| must be \texttt{True} for all positions \makebox{$(0, n)$, $0\leq n < \mathtt{height}$}
      \item \verb|North `elem` cells| must be \texttt{True} for all positions \makebox{$(n, \mathtt{height}-1)$, $0\leq n < \mathtt{width}$}
      \item \verb|East `elem` cells| must be \texttt{True} for all positions \makebox{$(\mathtt{width}-1, n)$, $0\leq n < \mathtt{height}$}
      \item \verb|West `elem` cells| must be \texttt{True} for all positions \makebox{$(n, 0)$, $0\leq n < \mathtt{width}$}
    \end{itemize}
  \item For a given position $(x,y)$ with corresponding cell $c$
    \begin{itemize}\setlength{\itemsep}{-2pt}
      \item If $\east \text{\func{`elem`} } c$ then the neighboring cell at $(x+1,y)$ must have a wall \west.
      \item If $\west \text{\func{`elem`} } c$ then the neighboring cell at $(x-1,y)$ must have a wall \east.
      \item If $\south \text{\func{`elem`} } c$ then the neighboring cell at $(x,y-1)$ must have a wall \north.
      \item If $\north \text{\func{`elem`} } c$ then the neighboring cell at $(x,y+1)$ must have a wall \north.
    \end{itemize}
  %% \item \east \el cells ! $(x, y)$ $\Leftrightarrow$ \west \el cells ! $(x+1, y)$ - One can only go from a to b if of can also go from b to a
  %% \item \north \el cells ! $(x, y)$ $\Leftrightarrow$\south \el cells ! $(x+1, y)$
  \item All cells within the given rectangle from $(0,0)$ to the goal
    position $(x,y)$ \emph{must} be specified.

    Comment: An interesting modification to this would be to assume that unspecified cells
    simply have no walls, except for cells on the border. Then a
    completely empty maze of size \src{width} and \src{height} could
    be specified as: \src{\scriptsize fromList [((width, height), [])]}
\end{itemize}

\subsection{\func{RobotCommand}}
For the type RobotCommand, we chose the following format:
\begin{lstlisting}[language=haskell]
newtype RobotCommand a = RC { runRC :: World -> Either Robot (a, Robot) }
\end{lstlisting}
A \func{RobotCommand} takes in a World, consisting of a \func{Maze} and a \func{Robot}. If the command is completed succesfully, it returns \func{Right} $(a, \text{\func{Robot}})$; if there is an error during the execution, it returns \func{Left Robot}, so that the position where the error has occurred and the robot's history up until that point can be retrieved. We only return a \func{Robot} (instead of a whole World) to reflect the fact that the maze cannot be changed through command execution.

In order to make this type a monadic instance, we created two functions:

\begin{lstlisting}[language=haskell]
instance Monad RobotCommand where
  return a     = RC $ \(_,r) -> Right (a, r)
  (RC h) >>= f = RC $ \w@(m,r) -> do (a, r') <- h w
                                     let (RC g) = f a
                                     g (m, r')
\end{lstlisting}

The bind operator appends one RobotCommand to another one by first running the first command on the input, and then running the second one on the state resulting from the first command. Note that, since the maze does not change and is not returned by the first RC, we run the second RC on a world consisting of our original maze and the robot returned by the first command.

Note that we use \func{do}-Syntax in the definition of our monad, which uses the \func{Either} monad and does the error handling for us.

The function \func{interp}, as specified in the assignment, turns a statement into a \func{RobotCommand}. The implementations for moving or changing direction simply manipulate the position and the direction of the input Robot. \func{evalCond} evaluates for all conditions if they are true in the given world. 

The implementations for \func{While} and \func{Block} commands might be of interest: The first evaluates the given condition and, if the condition is satisfied, executes a new block containing the loop body and the while statement itself; otherwise it just returns the current Robot. Similarly, the \func{Block} command executes the first statement in the block and then executes a new \func{Block} command containing the remaining block statements.

In order to run a given program, for example \textbf{wallFollowProg}, on a maze (e.g. \textbf{testMaze}), please run
\begin{lstlisting}[language=haskell]
  runProg testMaze wallFollowProg
\end{lstlisting} 

\section{Assessment}

%%Must contain: Assessment of the quality of code, and what the assessment is based on.

In general, our code does what it is intended to do. We have verified this as far as possible by a number of different unit tests (mostly black box):
\begin{itemize}
  \item There are a number of tests checking if statements are executed as expected. If not specified otherwise, they are all run in the provided \textbf{testMaze}. \textbf{testEmptyBlock} makes sure the execution of an empty block simply returns the world it has received.
  \item \textbf{testNavi} executes all basic movements like walking forward and backward and turning to both sides, and compares the resulting direction and history to the one we would expect from these statements.
  \item \textbf{testIf} runs a simple If command. It checks if the condition was evaluated correctly (our only white box test) and if the If command executes the correct statement. \textbf{testIf2} does the same but with a different condition, which should evaluate to false and therefore execute the other statement.
  \item \textbf{testWhile} tests a simple while command (going forward until a wall is reached) and makes sure the robot successfully stops at the right position.
  \item \textbf{testError} tries to execute a statement which would result in the robot jumping over a wall. It makes sure that this statement returns a failure with the correct history, and that subsequent statements (i.e. statements in a block coming after the one which fails) do not change the resulting position, direction or history.
\end{itemize}

In order to test the functionality of our code as a whole, we implemented a ``wall following'' algorithm for our robot.
For a simply connected maze (meaning that all walls are
connected to the edge) the robot is guaranteed to reach any cell (including the goal position) without getting lost by following the wall on its left (think of it as ``keeping one hand in contact with the left wall''). No maze
that is simply connected can contain a loop, so this algorithm
definitely solves this class of loop-less mazes. 

We checked if this algorithms works properly for three different cases:

\begin{itemize}
  \item For a one-cell maze, there is a test case \textbf{testOneCellMaze} which makes sure the program works correctly in a maze consisting of only one cell (i.e. it terminates immediately, since the robot is already at the goal position).
  \item Similarly, \textbf{testFourCellMaze} lets the robot find its goal in a simple $2 \times 2$ cells maze and makes sure it ends up at $(1,1)$.
  \item For the provided test maze, we used QuickCheck to implement a test (\textbf{testWallfollower}) which puts the robot on a random position in the maze, looking in a random direction. For all starting positions, the algorithm should now terminate successfully in the upper right corner. 
\end{itemize}

All tests (run \textbf{testAll}) return the expected result, or at least did so when we ran them (since it is possible that \textbf{testWallfollower} fails from a starting point QuickCheck did not try). We are therefore confident that our implementation agrees with the specification.

We assume that the starting point of the robot is always $(0,0)$ and that the goal is always the North-East corner of the maze. In general, we also expect the user to use \textbf{runProg} on a Maze and a Program whenever he wants to run a program. Other functions are exported solely for testing purposes. 

If runProg is \emph{not} used, it is possible to create an initial World with a Robot outside the borders of the maze. Running commands on this world will fail as soon as the maze is accessed. Since our feedback on the first submission indicated that checking our assumptions for the maze (see above) is unnecessary, we do not do so any more. Running programs on mazes which violate these constraints may therefore result in undefined behaviour.

It is not possible to build invalid statements, although it is possible to build an endless While loop, which, if interpreted, would result in an infinitely long RobotCommand. However, this is expected behaviour.

We are unaware of any glaring inefficiencies in our choice of data structures and algorithms; as stated above, we are quite happy with our implementation of cardinal and relative directions. We think Haskell's Data.Map type is a reasonable choice for the implementation of the maze structure itself, and otherwise implemented everything as specified. The only point where we are actually unsure if we have done the right thing is, as stated above, the implementation of the monad itself. 

\section{Extensions}
An interesting extension would be to allow the robot to inspect its
position. We could extend \func{Cond} in the following way:
\begin{lstlisting}[language=haskell]
data Cond = ...
          | Visited Position
\end{lstlisting}
which, when evaluated, inspects the history of the robot and returns
true if the given position is an element in the history. A possible implementation would look like this:
\begin{lstlisting}[language=haskell]
evalCond w@(m,r) cond = case cond of
  ...
  Visited pos -> pos `elem` (hist r)
\end{lstlisting}
Using this small extension, we could implement a depth-first search algorithm or
Trémaux' algorithm.

\section{On the resubmission}
We rewrote the calculation of absolute directions from cardinal directions and relative ones wo work without an enum. We no longer use any partial functions and make more use of the \textbf{RobotCommand} monad, so that our \textbf{interp} function is now entirely written in \textbf{do}-notation. We removed unnecessary checks of maze properties and added some test cases for other maze sizes (most importantly the one-cell maze). 



\end{document}
